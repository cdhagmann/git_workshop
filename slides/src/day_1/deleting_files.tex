% ======================================= %
% DELETING FILES
% ======================================= %

\section[Deleting]{Git Delete Commands}

\begin{frame}
\frametitle{\large Deleting a File (rm vs. git rm)}
\begin{itemize}
\item If you delete a file in your filesystem, you still need to commit your changes with \emph{git add file\_removed}.
\item Or you can use \emph{git rm file\_name}.
\end{itemize}
\end{frame}
\note{}

\begin{frame}
\frametitle{\large Deleting a File (rm vs. git rm)}
If you \emph{rm} a file, it will delete it locally, but it will still exist in your git directory. In order to fully delete a file, you need to use \emph{git rm}
\end{frame}
\note[itemize]{
\item More discussion at: \href{http://stackoverflow.com/questions/7434449/why-use-git-rm-to-remove-a-file-instead-of-rm}{http://stackoverflow.com/questions/7434449/why-use-git-rm-to-remove-a-file-instead-of-rm}
}

\begin{frame}
\frametitle{\large Deleting a File}
If you want to delete a file that has been staged, but not committed use:
\begin{itemize}
\item \emph{git rm -{}-cached}
\end{itemize}
\end{frame}
\note{}

\begin{frame}
\frametitle{\large Moving a File}
If you want to move a file use:
\begin{itemize}
\item \emph{git mv}
\end{itemize}
\end{frame}
\note{}

\begin{frame}
\frametitle{\large Discarding Changes to Unstaged Files}
If you want to discard changes to unstaged files use:
\begin{itemize}
\item \emph{git checkout -{}- filename}
\end{itemize}
Just keep in mind that branching is better practice...
\end{frame}
\note{}

\begin{frame}
\frametitle{\large Amending Staged Files}
In order to remove a file from the staged environment use:
\begin{itemize}
\item \emph{git reset filename}
\end{itemize}
\end{frame}
\note{}

\begin{frame}
\frametitle{\large Amending Existing Commits}
So you say you want to amend an existing commit? Why? \\
I purposely didn't add anything here. Don't do it...
\end{frame}
\note{}

\begin{frame}
\frametitle{\large Amending Commits}
Ok, fine...
\begin{itemize}
\item \emph{git commit -{}-amend}
\end{itemize}
\vspace{3mm}
But you are missing the point of version control...
\end{frame}
\note{}