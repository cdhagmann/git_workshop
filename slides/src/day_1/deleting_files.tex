% ======================================= %
% DELETING FILES
% ======================================= %

\section[Deleting]{Git Delete Commands}

\begin{frame}
\frametitle{\large Delete a File (rm vs. git rm)}
If you rm a file, it will delete it locally, but it will still exist in your git directory. In order to fully delete a file, you need to use git rm
\end{frame}
\note{}

\begin{frame}
\frametitle{\large Delete a File (rm vs. git rm)}
If you want to delete a file that has been staged, but not committed, use git rm -{}-cached
\end{frame}
\note{}

\begin{frame}
\frametitle{\large Delete a File (rm vs. git rm)}
If you want to move a file, use git mv
\end{frame}
\note{}

\begin{frame}
\frametitle{\large Discard Changes to Unstaged Files}
git checkout -{}- filename
Branching is better practice
\end{frame}
\note{}

\begin{frame}
\frametitle{\large Amending Staged Files}
In order to remove a file from the staged environment, use:
git reset filename
\end{frame}
\note{}

\begin{frame}
\frametitle{\large Amending Commits}
I purposely didn't add anything here. Don't do it...
\end{frame}
\note{}

\begin{frame}
\frametitle{\large Amending Commits}
Ok, fine...
git commit -{}-amend
\end{frame}
\note{}