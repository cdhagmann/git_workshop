% ======================================= %
% BASIC GIT WORK FLOW
% ======================================= %

\section[Basic Git]{Basic Git Work Flow}

\begin{frame}
\frametitle{\large The Basic Git Work Flow}
Files in your project can be in one of three states:
\begin{enumerate}
\item Modified
\item Staged
\item Committed
\end{enumerate}
\end{frame}
\note[itemize]{
\item Modified files have been changed on your computer, but they are not in the database yet.
\item Staged files means that you have tagged a modified file to be included in the next commit.
\item Committed files are safely stored in your local database.
}

\begin{frame}
\frametitle{\large The Basic Git Work Flow}
\begin{enumerate}
\item Synchronize Your Repo (git pull)
\item Make Changes to Your Code
\item Stage Changes for Commit (git add)
\item Commit Changes Locally (git commit)
\item Push Changes to Origin (git push)
\end{enumerate}
\end{frame}
\note{}

\begin{frame}
\frametitle{\large The Basic Git Work Flow}
In order to determine which files are in which state, you can use (most to least detail): \\
\emph{git diff} (unstaged changes only)\\
\emph{git status} \\
\emph{git status -s}
\end{frame}
\note{}

\begin{frame}
\frametitle{\large The Basic Git Work Flow}
In order to get a full history of your commits, you can use: \\
\emph{git log}
Everything is labeled with a SHA-1 checksum.
\end{frame}
\note{}

\begin{frame}
\frametitle{\large The Basic Git Work Flow}
In order to ignore certain files in your commits, you can change: \\
\emph{.gitignore}
There are lots of .gitignore templates online at https:// github.com/ github/ gitignore
Standard glob patterns work.
\end{frame}
\note{}

\begin{frame}
\frametitle{\large The Basic Git Work Flow}
Shortcuts: \\
\emph{git commit -m ``My message''}
Commit with a message.
\emph{git commit -a -m ``My message''}
Commit without staging with a message.
\end{frame}
\note{}