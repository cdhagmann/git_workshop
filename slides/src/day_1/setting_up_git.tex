% ======================================= %
% SETTING UP GIT
% ======================================= %

\section[Setting Up Git]{Setting Up Git On Your Machine}

\begin{frame}
\frametitle{\large{Setting Up Git - Linux}}
You can use the package management tool that comes with your distribution (use sudo):
\begin{enumerate}
\item yum install git
\item apt-get install git
\end{enumerate}
\end{frame}

\begin{frame}
\frametitle{\large{Setting Up Git - Mac}}
There are three main ways to install Git:
\begin{enumerate}
\item Install the Xcode Command Line Tools and Type ``git'' Into the Terminal
\item Binary Installer: \href{http://git-scm.com/download/mac}{http://git-scm.com/download/mac}
\item Git/GitHub GUI: \href{https://mac.github.com/}{https://mac.github.com/}
\end{enumerate}
\end{frame}

\begin{frame}
\frametitle{\large{Setting Up Git - Windows}}
There are three main ways to install Git:
\begin{enumerate}
\item Binary Installer: \href{http://git-scm.com/download/win}{http://git-scm.com/download/win}
\item msysGit: \href{http://msysgit.github.io/}{http://msysgit.github.io/}
\item Git/GitHub GUI: \href{https://windows.github.com/}{https://windows.github.com/}
\end{enumerate}
\end{frame}

\begin{frame}
\frametitle{\large{Setting Up a Git Repo}}
\begin{enumerate}
\item Create a New Directory (mkdir my-awesome-directory)
\item Navigate Into the Directory (cd my-awesome-directory)
\item Initialize the Directory (git init)
\end{enumerate}
\vspace{5mm}
The git init command creates a hidden directory called .git that contains all of the metadata for the project. \emph{You should never change anything in .git directly!}
\end{frame}
