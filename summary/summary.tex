\documentclass[11pt]{article}

\usepackage[utf8]{inputenc}
\usepackage{geometry}
\usepackage{graphicx}
\usepackage{amsmath}
\usepackage[parfill]{parskip}
\usepackage{booktabs}

%%%%%%%%%%%%%%%%%%%%%%%%%%%%%%

\title{Git Workshop Collaborated Summary}
\author{Everyone}
\date{October 30, 2014}

%%%%%%%%%%%%%%%%%%%%%%%%%%%%%%

\begin{document}

\maketitle
\tableofcontents
\pagebreak

%%%%%%%%%%%%%%%%%%%%%%%%%%%%%%

% INSTRUCTIONS: CREATE A NEW SECTION IN THE DOCUMENT AND WRITE YOUR PART OF THE SUMMARY BELOW IT

\section{Example Section}

This is an example of what you should do!

\section{\texttt{git fetch origin}}
The above command copies all branches from the remote refs/heads/ namespace and stores them to the local refs/remotes/origin/ namespace, unless the branch.$<$name$>$.fetch option is used to specify a non-default refspec.

\textbf{See also}: \texttt{git- fetch} Download objects and refs from another repository


\end{document}
